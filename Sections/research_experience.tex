\section{Research \\ Experience}

\subsection{\href{https://www.emory.edu/home/index.html}{Emory University}}[Atlanta, GA]
\vspace{-\parskip}
Ph.D. Candidate (Dissertation) \hfill \printdate{Aug 2022~--~Present} \\
    \begin{description}[leftmargin=!,labelwidth=1.0em,labelindent=1.0em]
    %\item[Title:] Bayesian Tree-Based Methods for Environmental Health Research
    \item[Advisor:] Dr. Howard Chang
    \item[Description:] Extends the Bayesian additive regression trees (BART) framework to:
    \begin{itemize}
        \item Heterogeneous heat wave effect estimation for Alzheimer's disease patients within the case-crossover design,
        \item Smooth exposure-risk surface estimation for air pollution mixtures and asthma, and
        \item A spatially-varying quantile G-computation approach for estimating effects of air pollution mixtures on infant birth weight.
    \end{itemize}
    \item[Methods:] BART, Markov chain Monte Carlo (MCMC), conditional logistic regression, negative binomial regression, conditional autoregressive models, quantile G-computation
    \end{description}

Research Assistant \hfill \printdate{Dec 2021~--~Present} \\
    \begin{description}[leftmargin=!,labelwidth=1.0em,labelindent=1.0em]
    \item[Title:] Sharing Patients' Illness Representations to Increase Trust (SPIRIT)
    \item[Advisors:] Dr. Amita Manatunga and Dr. Mi-Kyung Song
    \item[Description:] Analyzed data from a multi-site cluster randomized trial to assess the efficacy of an intervention to improve decision making confidence, post-bereavement outcomes, and treatment intensity for patients with end-stage renal disease and their surrogates. 
    \end{description}
    \vspace{-\parskip}
    \indent \quad Methods: Generalized linear mixed models, generalized estimating equations

Research Assistant \hfill \printdate{May 2021~--~Dec 2021} \\
    \begin{description}[leftmargin=!,labelwidth=1.0em,labelindent=1.0em]
    \item[Advisor:] Dr. Lance Waller
    \item[Description:] Investigated the ability of sequentially layered spatial smoothing and spatial cluster detection techniques to identify hot spots for opioid overdoses in Georgia.
    \item[Methods:] Inverse-distance smoothing, Besag-York-Mollié model, integrated nested Laplace approximation (INLA); clustering tests of Turnbull, Besag \& Newell, and Kulldorff
    \end{description}

\subsection{\href{https://www.nku.edu}{Northern Kentucky University}}[Highland Heights, KY]
\vspace{-\parskip}
Undergraduate Research Assistant \hfill \printdate{Jan 2018~--~May 2019} \\
    \begin{description}[leftmargin=!,labelwidth=1.0em,labelindent=1.0em]
    \item[Advisor:] Dr. Andrew Long
    \item[Description:] Collaborated with Togolese meteorologists to estimate long-term trends and seasonality in temperature time series data measured across 10 Togolese cities.
    \item[Methods:] Singular spectrum analysis, linear mixed models
    \end{description}
    
Undergraduate Research Assistant \hfill \printdate{Jan 2017~--~May 2017} \\
    \begin{description}[leftmargin=!,labelwidth=1.0em,labelindent=1.0em]
    \item[Advisors:] Dr. Dhanuja Kasturiratna, Dr. Lisa Holden, and Dr. Stuart Goldstein
    \item[Description:] Collaborated with Cincinnati Children’s Hospital to develop a model to predict the development of chronic kidney disease in children with acute kidney injury.
    \item[Methods:] Logistic regression
    \end{description}


    